  \section[Architectures cloud]{Architectures cloud-ready}

  \subsection[Architecture dev]{Concevoir une application pour le cloud}

  \begin{frame}
    \frametitle{Risques des applications legacy}

  \end{frame}

  \begin{frame}
    \frametitle{12-factor}
    "The Twelve-Factor App" \url{http://12factor.net/}
    \begin{itemize}
      \item Écrit par Heroku
      \item Suivre (tout) le code dans un VCS
      \item Configuration
    \end{itemize}
  \end{frame}

  \begin{frame}
    \frametitle{Modulaire}
    \begin{itemize}
      \item Multiples composants de taille raisonnable
      \item Philosophie Unix
      \item Couplage faible et interface documentée
    \end{itemize}
  \end{frame}

  \begin{frame}
    \frametitle{Passage à l'échelle}
    \begin{itemize}
      \item Vertical vs Horizontal
      \item Plusieurs petites instances plutôt qu'une grosse instance
    \end{itemize}
  \end{frame}

  \begin{frame}
    \frametitle{Stateful vs stateless}
    \begin{itemize}
      \item Beaucoup de stateful dans les applications legacy
      \item Nécessite de partager l'information d'état lorsque plusieurs workers
      \item Le stateless élimine cette contrainte
    \end{itemize}
  \end{frame}

  \begin{frame}
    \frametitle{Tolérance aux pannes}
    \begin{itemize}
      \item L'infrastructure n'est pas hautement disponible
      \item L'API d'infrastructure est hautement disponible
      \item L'application doit anticiper et réagir aux pannes
    \end{itemize}
  \end{frame}

  \begin{frame}
    \frametitle{Stockage des données}
    \begin{itemize}
      \item Base de données relationnelles
      \item Base de données NoSQL
      \item Stockage bloc
      \item Stockage objet
      \item Stockage éphémère
      \item Cache, temporaire
    \end{itemize}
  \end{frame}

  \begin{frame}
    \frametitle{Design Tenets d'OpenStack (exemple)}
    \begin{enumerate}
      \item Scalability and elasticity are our main goals
      \item Any feature that limits our main goals must be optional
      \item Everything should be asynchronous. If you can't do something asynchronously, see \#2
      \item All required components must be horizontally scalable
      \item Always use shared nothing architecture (SN) or sharding. If you can't Share nothing/shard, see \#2
      \item Distribute everything. Especially logic. Move logic to where state naturally exists.
      \item Accept eventual consistency and use it where it is appropriate.
      \item Test everything. We require tests with submitted code. (We will help you if you need it)
    \end{enumerate}
    \url{https://wiki.openstack.org/wiki/BasicDesignTenets}
  \end{frame}


  \begin{frame}[allowframebreaks]
    \frametitle{Adapter ou penser ses applications "cloud ready"}
    Cf. les design tenets du projet OpenStack et Twelve-Factor \url{http://12factor.net/}
    \begin{itemize}
      \item Architecture distribuée plutôt que monolithique
      \begin{itemize}
        \item Facilite le passage à l'échelle
        \item Limite les domaines de \textit{failure}
      \end{itemize}\pause
      \item Couplage faible entre les composants
      \item Bus de messages pour les communications inter-composants\framebreak
      \item Stateless : permet de multiplier les routes d'accès à l'application\pause
      \item Dynamicité : l'application doit s'adapter à son environnement et se reconfigurer lorsque nécessaire\pause
      \item Permettre le déploiement et l'exploitation par des outils d'automatisation\pause
      \item Limiter autant que possible les dépendances à du matériel ou du logiciel spécifique qui pourrait ne pas fonctionner dans un cloud\pause
      \item Tolérance aux pannes (\textit{fault tolerance}) intégrée\pause
      \item Ne pas stocker les données en local, mais plutôt :
      \begin{itemize}
        \item Base de données
        \item Stockage objet
      \end{itemize}\pause
      \item Utiliser des outils standards de journalisation
    \end{itemize}
  \end{frame}
