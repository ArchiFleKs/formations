  \subsection[Architecture infra]{Concevoir une infrastructure pour le cloud}

  \begin{frame}
    \frametitle{Adopter une philosophie DevOps}
    \begin{itemize}
      \item Infrastructure as Code
      \item Scale out plutôt que scale up (horizontalement plutôt que verticalement)
      \item HA niveau application plutôt qu'infrastructure
      \item Automatisation, automatisation, automatisation
      \item Tests
    \end{itemize}
  \end{frame}

  \begin{frame}
    \frametitle{Automatisation}
    \begin{itemize}
      \item Automatiser la gestion de l'infrastructure : indispensable
      \item Création des ressources
      \item Configuration des ressources
    \end{itemize}
  \end{frame}

  \begin{frame}
    \frametitle{Infrastructure as Code}
    \begin{itemize}
      \item Travailler comme un développeur
      \item Décrire son infrastructure sous forme de code (Heat, Ansible)
      \item Suivre les changements dans un VCS (git)
      \item Utiliser des outils de tests
    \end{itemize}
  \end{frame}

  \begin{frame}
    \frametitle{Besoin d'orchestration}
    \begin{itemize}
      \item Manager tous les types de ressources par un point d'entrée
      \item Description de l'infrastructure dans un fichier (\textit{template})
      \item Heat (intégré à OpenStack), Terraform
    \end{itemize}
  \end{frame}

  \begin{frame}
    \frametitle{Tests et intégration continue}
    \begin{itemize}
      \item Style de code
      \item Validation de la syntaxe
    \end{itemize}
  \end{frame}

  \begin{frame}
    \frametitle{L'isolation}
    \begin{itemize}
      \item Niveau control plane : Tenant (projet)
      \item Niveau réseau : L2, L3, security groups
    \end{itemize}
  \end{frame}

  \begin{frame}
    \frametitle{Réseau}
    \begin{itemize}
      \item Fixed IP
      \item Multiples interfaces réseaux
      \item Floating IPs : pool, allocate, associate
    \end{itemize}
  \end{frame}

  \begin{frame}
    \frametitle{Les instances}
    \begin{itemize}
      \item Éphémère
      \item Pets vs Cattle
      \item Basé sur une \textit{image}
      \item API de metadata
    \end{itemize}
  \end{frame}

  \begin{frame}
    \frametitle{Monitoring}
    Monitoring
    \begin{itemize}
      \item Prendre en compte le cycle de vie des instances : DOWN != ALERT
      \item Monitorer le service plus que le serveur
    \end{itemize}
  \end{frame}

  \begin{frame}
    \frametitle{Backup}
    Backuper, quoi ?
    \begin{itemize}
      \item Être capable de recréer ses instances (et le reste de son infrastructure)
      \item Données (applicatives, logs) : block, objet
    \end{itemize}
  \end{frame}

  \begin{frame}
    \frametitle{Un exemple : l'équipe openstack-infra}
    \begin{itemize}
      \item Infrastructure as code
      \item Infrastructure ouverte : code "open source"
      \item Utilise du cloud (hybride)
    \end{itemize}
  \end{frame}
